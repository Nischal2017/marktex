% Options for packages loaded elsewhere
\PassOptionsToPackage{unicode}{hyperref}
\PassOptionsToPackage{hyphens}{url}
%
\documentclass[
]{article}
\usepackage{amsmath,amssymb}
\usepackage{iftex}
\ifPDFTeX
  \usepackage[T1]{fontenc}
  \usepackage[utf8]{inputenc}
  \usepackage{textcomp} % provide euro and other symbols
\else % if luatex or xetex
  \usepackage{unicode-math} % this also loads fontspec
  \defaultfontfeatures{Scale=MatchLowercase}
  \defaultfontfeatures[\rmfamily]{Ligatures=TeX,Scale=1}
\fi
\usepackage{lmodern}
\ifPDFTeX\else
  % xetex/luatex font selection
\fi
% Use upquote if available, for straight quotes in verbatim environments
\IfFileExists{upquote.sty}{\usepackage{upquote}}{}
\IfFileExists{microtype.sty}{% use microtype if available
  \usepackage[]{microtype}
  \UseMicrotypeSet[protrusion]{basicmath} % disable protrusion for tt fonts
}{}
\makeatletter
\@ifundefined{KOMAClassName}{% if non-KOMA class
  \IfFileExists{parskip.sty}{%
    \usepackage{parskip}
  }{% else
    \setlength{\parindent}{0pt}
    \setlength{\parskip}{6pt plus 2pt minus 1pt}}
}{% if KOMA class
  \KOMAoptions{parskip=half}}
\makeatother
\usepackage{xcolor}
\usepackage{color}
\usepackage{fancyvrb}
\newcommand{\VerbBar}{|}
\newcommand{\VERB}{\Verb[commandchars=\\\{\}]}
\DefineVerbatimEnvironment{Highlighting}{Verbatim}{commandchars=\\\{\}}
% Add ',fontsize=\small' for more characters per line
\newenvironment{Shaded}{}{}
\newcommand{\AlertTok}[1]{\textcolor[rgb]{1.00,0.00,0.00}{\textbf{#1}}}
\newcommand{\AnnotationTok}[1]{\textcolor[rgb]{0.38,0.63,0.69}{\textbf{\textit{#1}}}}
\newcommand{\AttributeTok}[1]{\textcolor[rgb]{0.49,0.56,0.16}{#1}}
\newcommand{\BaseNTok}[1]{\textcolor[rgb]{0.25,0.63,0.44}{#1}}
\newcommand{\BuiltInTok}[1]{\textcolor[rgb]{0.00,0.50,0.00}{#1}}
\newcommand{\CharTok}[1]{\textcolor[rgb]{0.25,0.44,0.63}{#1}}
\newcommand{\CommentTok}[1]{\textcolor[rgb]{0.38,0.63,0.69}{\textit{#1}}}
\newcommand{\CommentVarTok}[1]{\textcolor[rgb]{0.38,0.63,0.69}{\textbf{\textit{#1}}}}
\newcommand{\ConstantTok}[1]{\textcolor[rgb]{0.53,0.00,0.00}{#1}}
\newcommand{\ControlFlowTok}[1]{\textcolor[rgb]{0.00,0.44,0.13}{\textbf{#1}}}
\newcommand{\DataTypeTok}[1]{\textcolor[rgb]{0.56,0.13,0.00}{#1}}
\newcommand{\DecValTok}[1]{\textcolor[rgb]{0.25,0.63,0.44}{#1}}
\newcommand{\DocumentationTok}[1]{\textcolor[rgb]{0.73,0.13,0.13}{\textit{#1}}}
\newcommand{\ErrorTok}[1]{\textcolor[rgb]{1.00,0.00,0.00}{\textbf{#1}}}
\newcommand{\ExtensionTok}[1]{#1}
\newcommand{\FloatTok}[1]{\textcolor[rgb]{0.25,0.63,0.44}{#1}}
\newcommand{\FunctionTok}[1]{\textcolor[rgb]{0.02,0.16,0.49}{#1}}
\newcommand{\ImportTok}[1]{\textcolor[rgb]{0.00,0.50,0.00}{\textbf{#1}}}
\newcommand{\InformationTok}[1]{\textcolor[rgb]{0.38,0.63,0.69}{\textbf{\textit{#1}}}}
\newcommand{\KeywordTok}[1]{\textcolor[rgb]{0.00,0.44,0.13}{\textbf{#1}}}
\newcommand{\NormalTok}[1]{#1}
\newcommand{\OperatorTok}[1]{\textcolor[rgb]{0.40,0.40,0.40}{#1}}
\newcommand{\OtherTok}[1]{\textcolor[rgb]{0.00,0.44,0.13}{#1}}
\newcommand{\PreprocessorTok}[1]{\textcolor[rgb]{0.74,0.48,0.00}{#1}}
\newcommand{\RegionMarkerTok}[1]{#1}
\newcommand{\SpecialCharTok}[1]{\textcolor[rgb]{0.25,0.44,0.63}{#1}}
\newcommand{\SpecialStringTok}[1]{\textcolor[rgb]{0.73,0.40,0.53}{#1}}
\newcommand{\StringTok}[1]{\textcolor[rgb]{0.25,0.44,0.63}{#1}}
\newcommand{\VariableTok}[1]{\textcolor[rgb]{0.10,0.09,0.49}{#1}}
\newcommand{\VerbatimStringTok}[1]{\textcolor[rgb]{0.25,0.44,0.63}{#1}}
\newcommand{\WarningTok}[1]{\textcolor[rgb]{0.38,0.63,0.69}{\textbf{\textit{#1}}}}
\usepackage{longtable,booktabs,array}
\usepackage{calc} % for calculating minipage widths
% Correct order of tables after \paragraph or \subparagraph
\usepackage{etoolbox}
\makeatletter
\patchcmd\longtable{\par}{\if@noskipsec\mbox{}\fi\par}{}{}
\makeatother
% Allow footnotes in longtable head/foot
\IfFileExists{footnotehyper.sty}{\usepackage{footnotehyper}}{\usepackage{footnote}}
\makesavenoteenv{longtable}
\usepackage{graphicx}
\makeatletter
\def\maxwidth{\ifdim\Gin@nat@width>\linewidth\linewidth\else\Gin@nat@width\fi}
\def\maxheight{\ifdim\Gin@nat@height>\textheight\textheight\else\Gin@nat@height\fi}
\makeatother
% Scale images if necessary, so that they will not overflow the page
% margins by default, and it is still possible to overwrite the defaults
% using explicit options in \includegraphics[width, height, ...]{}
\setkeys{Gin}{width=\maxwidth,height=\maxheight,keepaspectratio}
% Set default figure placement to htbp
\makeatletter
\def\fps@figure{htbp}
\makeatother
\ifLuaTeX
  \usepackage{luacolor}
  \usepackage[soul]{lua-ul}
\else
  \usepackage{soul}
\fi
\setlength{\emergencystretch}{3em} % prevent overfull lines
\providecommand{\tightlist}{%
  \setlength{\itemsep}{0pt}\setlength{\parskip}{0pt}}
\setcounter{secnumdepth}{-\maxdimen} % remove section numbering
\ifLuaTeX
  \usepackage{selnolig}  % disable illegal ligatures
\fi
\IfFileExists{bookmark.sty}{\usepackage{bookmark}}{\usepackage{hyperref}}
\IfFileExists{xurl.sty}{\usepackage{xurl}}{} % add URL line breaks if available
\urlstyle{same}
\hypersetup{
  hidelinks,
  pdfcreator={LaTeX via pandoc}}

\author{}
\date{}

\begin{document}

\hypertarget{marktex-example-document}{%
\section{MarkTeX Example Document}\label{marktex-example-document}}

This is a sample document demonstrating MarkTeX's ability to convert
Markdown with Mermaid diagrams to beautiful PDFs.

\hypertarget{introduction}{%
\subsection{Introduction}\label{introduction}}

MarkTeX combines the simplicity of Markdown with the power of LaTeX and
the clarity of Mermaid diagrams. This makes it perfect for:

\begin{itemize}
\tightlist
\item
  Technical documentation
\item
  Research reports
\item
  Project proposals
\item
  Academic papers
\item
  System design documents
\end{itemize}

\hypertarget{workflow-diagram}{%
\subsection{Workflow Diagram}\label{workflow-diagram}}

Here's a simple workflow showing how MarkTeX processes your documents:

\includegraphics{mermaid-images/c705c5fba66d4d0c1a54407a931f9e69e9c96408.png}

\hypertarget{system-architecture}{%
\subsection{System Architecture}\label{system-architecture}}

Complex systems are easier to understand with diagrams:

\includegraphics{mermaid-images/98809a8c8656af4d91f7b9ee4f91a7da7b5622ec.png}

\hypertarget{process-flow}{%
\subsection{Process Flow}\label{process-flow}}

\includegraphics{mermaid-images/603a96fa86a82f4986f104029d1727ce4cd455be.png}

\hypertarget{code-example}{%
\subsection{Code Example}\label{code-example}}

MarkTeX also handles code blocks beautifully:

\begin{Shaded}
\begin{Highlighting}[]
\KeywordTok{def}\NormalTok{ convert\_markdown\_to\_pdf(input\_file):}
    \CommentTok{"""Convert a Markdown file to PDF using MarkTeX."""}
\NormalTok{    subprocess.run([}
        \StringTok{"marktex"}\NormalTok{,}
\NormalTok{        input\_file}
\NormalTok{    ])}
    \BuiltInTok{print}\NormalTok{(}\SpecialStringTok{f"✓ Converted }\SpecialCharTok{\{}\NormalTok{input\_file}\SpecialCharTok{\}}\SpecialStringTok{ to PDF!"}\NormalTok{)}
\end{Highlighting}
\end{Shaded}

\hypertarget{features}{%
\subsection{Features}\label{features}}

\hypertarget{text-formatting}{%
\subsubsection{Text Formatting}\label{text-formatting}}

You can use \textbf{bold}, \emph{italic}, \texttt{inline\ code}, and
even \st{strikethrough} text.

\hypertarget{lists}{%
\subsubsection{Lists}\label{lists}}

Ordered lists:

\begin{enumerate}
\def\labelenumi{\arabic{enumi}.}
\tightlist
\item
  First item
\item
  Second item
\item
  Third item
\end{enumerate}

Unordered lists:

\begin{itemize}
\tightlist
\item
  Feature A
\item
  Feature B

  \begin{itemize}
  \tightlist
  \item
    Sub-feature B.1
  \item
    Sub-feature B.2
  \end{itemize}
\item
  Feature C
\end{itemize}

\hypertarget{tables}{%
\subsubsection{Tables}\label{tables}}

\begin{longtable}[]{@{}llll@{}}
\toprule\noalign{}
Feature & Markdown & LaTeX & PDF \\
\midrule\noalign{}
\endhead
\bottomrule\noalign{}
\endlastfoot
Text & ✓ & ✓ & ✓ \\
Images & ✓ & ✓ & ✓ \\
Diagrams & ✓ & ✓ & ✓ \\
Math & ✓ & ✓ & ✓ \\
\end{longtable}

\hypertarget{mathematics}{%
\subsubsection{Mathematics}\label{mathematics}}

Inline math: \(E = mc^2\)

Block math:

\[
\int_{-\infty}^{\infty} e^{-x^2} dx = \sqrt{\pi}
\]

\hypertarget{sequence-diagram}{%
\subsection{Sequence Diagram}\label{sequence-diagram}}

\includegraphics{mermaid-images/939b21d12f0f5f346fd9164fe14269226bb0d012.png}

\hypertarget{state-diagram}{%
\subsection{State Diagram}\label{state-diagram}}

\includegraphics{mermaid-images/751290d7db18cecb61d4446dc8c65219f4a59037.png}

\hypertarget{conclusion}{%
\subsection{Conclusion}\label{conclusion}}

MarkTeX makes it easy to create professional documents with diagrams.
Simply write your content in Markdown, add Mermaid diagrams where
needed, and run:

\begin{Shaded}
\begin{Highlighting}[]
\ExtensionTok{marktex}\NormalTok{ example.md}
\end{Highlighting}
\end{Shaded}

That's it! You'll get a beautiful PDF with all your diagrams rendered
perfectly.

\hypertarget{learn-more}{%
\subsection{Learn More}\label{learn-more}}

\begin{itemize}
\tightlist
\item
  GitHub: https://github.com/yourusername/marktex
\item
  Documentation: See README.md
\item
  License: MIT
\end{itemize}

\begin{center}\rule{0.5\linewidth}{0.5pt}\end{center}

\emph{Generated with MarkTeX - Markdown to PDF made simple. Now with
Tectonic!}

\end{document}
